\documentclass[conference]{IEEEtran}
% \IEEEoverridecommandlockouts
% The preceding line is only needed to identify funding in the first footnote. If that is unneeded, please comment it out.
%Template version as of 6/27/2024

\usepackage[slovak]{babel}
\usepackage{cite}
\usepackage{amsmath,amssymb,amsfonts}
\usepackage{algorithmic}
\usepackage{graphicx}
\usepackage{textcomp}
\usepackage{xcolor}
\usepackage{url}
\usepackage{hyperref}
\def\BibTeX{{\rm B\kern-.05em{\sc i\kern-.025em b}\kern-.08em
    T\kern-.1667em\lower.7ex\hbox{E}\kern-.125emX}}
\begin{document}

\title{Balík nástrojov Microsoft Sysinternals}

\author{\IEEEauthorblockN{Marek Čederle}
\IEEEauthorblockA{\textit{Fakulta informatiky a informačných technológií} \\
\textit{Slovenská Technická Univerzita v Bratislave}\\
Bratislava, Slovensko \\
xcederlem@stuba.sk}
}

\maketitle

\begin{abstract}
TODO: WRITE AN ABSTRACT
This document is a model and instructions for \LaTeX.
This and the IEEEtran.cls file define the components of your paper [title, text, heads, etc.]. *CRITICAL: Do Not Use Symbols, Special Characters, Footnotes, 
or Math in Paper Title or Abstract.
\end{abstract}

\begin{IEEEkeywords}
TODO: INSERT KEYWORDS, component, formatting, style, styling, insert.
\end{IEEEkeywords}


\nocite{*}

\section{Funkčný opis bezpečnostného nástroja}
\subsection{Úvod do Microsoft Sysinternals}
% Stručná história a pozadie balíka MS Sysinternals. Účel a ciele nástrojov Sysinternals, so zameraním na monitorovanie, diagnostiku a riešenie problémov v systémoch Windows.

Windows Sysinternals je balík nástrojov, ktorý sa zameriava na správu, diagnostiku, riešenie problémov a monitorovanie operačného systému MS\footnote{Microsoft} Windows. Pôvodne Sysinternals bola webová stránka (predtým známa ako ntinternals) vytvorená v roku 1996 a prevádzkovala ju spoločnosť Winternals Software LP so sídlom v Austine v Texase. Založili ju softvéroví vývojári Bryce Cogswell a Mark Russinovich. 18. júla 2006 spoločnosť Winternals bola zakúpená spoločnosťou Microsoft.

Webová stránka obsahovala niekoľko freewarových nástrojov na správu a monitorovanie počítačov s operačným systémom MS Windows. Spoločnosť predávala aj nástroje na obnovu dát a profesionálne edície svojich freewarových nástrojov.
Aktuálny zoznam nástrojov je zobrazený v ``Tabuľke č.~\ref{tab1}''.

\begin{table}[htbp]
    \caption{Zoznam nástrojov}
    \begin{center}
    \begin{tabular}{|c|c|c|}
    \hline
    AccessChk & Junction & PsService \\ \hline
    AccessEnum & LDMDump & PsShutdown \\ \hline
    AdExplorer & ListDLLs & PsSuspend \\ \hline
    AdInsight & LiveKd & PsTools \\ \hline
    AdRestore & LoadOrder & RAMMap \\ \hline
    Autologon & LogonSessions & RDCMan \\ \hline
    \textbf{Autoruns} & MoveFile & RegDelNull \\ \hline
    BgInfo & NotMyFault & RegHide \\ \hline
    BlueScreen & NTFSInfo & RegJump \\ \hline
    CacheSet & PendMoves & Registry Usage (RU) \\ \hline
    ClockRes & PipeList & SDelete \\ \hline
    Contig & PortMon & ShareEnum \\ \hline
    Coreinfo & ProcDump & ShellRunas \\ \hline
    Ctrl2Cap & \textbf{Process Explorer} & Sigcheck \\ \hline
    DebugView & \textbf{Process Monitor} & Streams \\ \hline
    Desktops & PsExec & Strings \\ \hline
    Disk2vhd & PsFile & Sync \\ \hline
    DiskExt & PsGetSid & Sysmon \\ \hline
    DiskMon & PsInfo & \textbf{TCPView} \\ \hline
    DiskView & PsKill & VMMap \\ \hline
    Disk Usage (DU) & PsList & VolumeID \\ \hline
    EFSDump & PsLoggedOn & WhoIs \\ \hline
    FindLinks & PsLogList & WinObj \\ \hline
    Handle & PsPasswd & ZoomIt \\ \hline
    Hex2dec & PsPing & \\ \hline
    \end{tabular}
    \label{tab1}
    \end{center}
\end{table}


\subsection{Prehľad vybraných nástrojov}
Pre účely tohto projektu som sa zameral iba na pár vybraných nástrojov z balíku nástrojov Sysinternals, ktoré sú zamerané na bezpečnosť alebo majú uplatnenie v tejto oblasti. Je to predovšetkým kvôli tomu, že Sysinternals obsahuje nesmierne veľa rôznych nástrojov a bolo by to mimo rozsahu a zamerania tejto práce v predmete PRBIT\footnote{Princípy bezpečnosti v informačných technológiách}.
\subsubsection{Process Explorer}
% \begin{itemize}
%     \item Čo je Process Explorer?
%     \item Hlavné funkcie: monitorovanie systému, vizualizácia stromu procesov, analýza DLL, atď.
% \end{itemize}
Process explorer je nástroj, ktorý má veľmi podobné funkcionality ako klasický Task Manager\footnote{Správca úloh}, ktorý je vstavaný v každom operačnom systéme MS Windows už od verzie Windows NT 4.0. Vie zobraziť využite CPU\footnote{procesora} procesmi na danom systéme. Taktiež využitie operačnej pamäte jednotlivými procesmi, ich PID\footnote{Process ID} a popis. Jeho výhodou je, že je veľmi podrobný oproti jednoduchému Správcovi úloh, hlavne čo sa týka informácií ohľadom správy pamäte. Zobrazuje procesy v tzv. stromovej štruktúre čo znamená, že je jednoducho vidieť, ktorý proces spustil iný proces (resp. ktorý proces je rodičom Ďalších procesov). Po kliknutí na vybraný proces, program dokáže zobraziť vlákna\footnote{threads} daného procesu a dynamické knižnice\footnote{DLL --- Dynamic-Link Library}, ktoré daný proces používa. Po otvorení vlastností jednotlivého procesu nám vyskočí okno, kde sú všetky podrobnosti, ktoré sa vôbec o procese v rámci operačného systému dajú získať. Nástroj obsahuje aj vyhľadávanie procesov, čo bola funckionalita, ktorú napríklad Správca úloh získal až vo verzii Windows 11. Dalšou funkciou Process Explorera je aj možnosť zobraziť informácie o sieťových pripojeniach, ktoré daný proces vytvára. Taktiež dokáže zobraziť bezpečnostné politiky, ktoré boli aplikovaný na daný proces. V neposlednom rade tento nástroj vie poslať vzorku daného procesu (programu) na webovú stránku \href{https://www.virustotal.com/gui/home/upload}{VirusTotal.com} ktorá slúži na skenovanie potenciálne škodlivého softvéru\footnote{malware}. 

% TODO: citovat niekde priamo aj tie zdroje

\subsubsection{AutoRuns}
\begin{itemize}
    \item Účel nástroja AutoRuns: správa programov spúšťaných pri štarte systému.
    \item Kľúčové funkcie: identifikácia programov pri štarte, overenie podpísaných súborov, kontrola naplánovaných úloh.
\end{itemize}

\subsubsection{TCPView}
\begin{itemize}
    \item Úloha pri monitorovaní sieťovej aktivity.
    \item Funkcie: zobrazenie sieťových pripojení, riešenie DNS, monitorovanie TCP/UDP prevádzky.
\end{itemize}

\subsubsection{Process Monitor}
\begin{itemize}
    \item Čo robí: monitorovanie súborového systému, registrov a procesov v reálnom čase.
    \item Zaujímavé funkcie: filtrovanie udalostí, sledovanie systémových volaní, hľadanie neoprávnených zmien.
\end{itemize}

\subsection{Porovnanie s Inými Bezpečnostnými Nástrojmi}
Ako sa nástroje Sysinternals porovnávajú s inými populárnymi bezpečnostnými balíkmi (napr. Process Hacker, Wireshark, GlassWire)? Výhody a nevýhody.

\section{Požiadavky na inštaláciu a postup pri inštalácii}
\subsection{Systémové Požiadavky}
\begin{itemize}
    \item Podporované operačné systémy.
    \item Hardvérové požiadavky (ak nejaké sú).
    \item Predpoklady (napr. používateľské oprávnenia, práva správcu).
\end{itemize}

\subsection{Inštalačný Proces}
\subsubsection{Sťahovanie Nástrojov}
\begin{itemize}
    \item Kde sťahovať jednotlivé nástroje (oficiálna stránka Microsoftu Sysinternals).
    \item Formáty súborov (ZIP, EXE, atď.) a veľkosť.
\end{itemize}

\subsubsection{Inštalácia/Spustenie}
\begin{itemize}
    \item Krok za krokom návod na spustenie a inštaláciu jednotlivých nástrojov.
    \item Vysvetlenie samostatných nástrojov (netreba inštalovať, len spustiť).
    \item Rozdiely medzi spúšťaním nástrojov lokálne a zo sieťového zdieľania.
\end{itemize}

\subsection{Konfigurácia po Inštalácii}
\begin{itemize}
    \item Úvodné nastavenia a konfigurácie pre jednotlivé nástroje.
    \item Nastavenie prostredia pre monitorovanie a diagnostiku.
\end{itemize}

\section{Experimentovanie a overeniu základných funkcionalít bezpečnostného nástroja}
\subsection{Príprava na Experimentovanie}
\begin{itemize}
    \item Prehľad testovacieho prostredia: verzia OS, virtuálny stroj (ak sa použil), základná konfigurácia.
    \item Popis scenára alebo prípadu použitia: príklady scenárov pre testovanie každého nástroja.
\end{itemize}

\subsection{Experimenty s Konkrétnymi Nástrojmi}
\subsubsection{Process Explorer}
\begin{itemize}
    \item Analýza bežiacich procesov a identifikácia anomálií.
    \item Použitie vyhľadávacích funkcií na sledovanie konkrétnych súborov alebo procesov.
    \item Experimentovanie s integráciou "VirusTotal".
\end{itemize}

\subsubsection{AutoRuns}
\begin{itemize}
    \item Zoznam všetkých programov a služieb spúšťaných pri štarte.
    \item Povolenie/zablokovanie položiek a kontrola vplyvu na správanie systému.
    \item Overovanie digitálnych podpisov spúšťacích položiek.
\end{itemize}

\subsubsection{TCPView}
\begin{itemize}
    \item Monitorovanie živých sieťových pripojení.
    \item Identifikácia nezvyčajných vzdialených pripojení.
    \item Analýza stavov pripojenia (napr. TIME\_WAIT, LISTENING).
\end{itemize}

\subsubsection{Process Monitor}
\begin{itemize}
    \item Filtrovanie konkrétnych registrov, súborového systému alebo sieťových udalostí.
    \item Vytváranie filtrov udalostí na zachytenie cieľových aktivít.
    \item Identifikácia zmien v registroch alebo súboroch počas inštalácií softvéru.
\end{itemize}

\section{Dokumentovanie experimentovania s nástrojom formou napríklad riadacich príkazov alebo obrazoviek alebo nastavení}
\subsection{Podrobné Správy z Experimentov}
\begin{itemize}
    \item Screenshoty zobrazujúce kroky a výsledky každého experimentu.
    \item Príklady príkazov alebo konfigurácií použitých v experimentoch.
\end{itemize}

\subsection{Analýza Záznamov a Výstupov}
\begin{itemize}
    \item Prezentácia záznamov alebo dát zhromaždených pomocou každého nástroja.
    \item Ako interpretovať zhromaždené informácie.
    \item Porovnanie očakávaných a pozorovaných výsledkov.
\end{itemize}

\subsection{Problémy, na ktoré sa Narazilo, a Riešenia}
\begin{itemize}
    \item Dokumentovanie akýchkoľvek problémov alebo chýb počas experimentov.
    \item Kroky na vyriešenie problémov alebo úpravu nastavení.
\end{itemize}

\section{Hodnotenie bezpečnostného nástroja}
\subsection{Účinnosť a Spoľahlivosť}
Ako dobre si nástroje viedli vo vašich experimentoch? Zaznamenané silné stránky alebo slabiny.

\subsection{Jednoduchosť Použitia a Užívateľské Rozhranie}
Ako intuitívne sú rozhrania pre jednotlivé nástroje? Prístupnosť pre začiatočníkov a pokročilých používateľov.

\subsection{Vplyv na Výkon}
Vplyv spustenia nástrojov na výkon systému. Zaznamenané spomalenia alebo špičky vo využití zdrojov.

\subsection{Bezpečnosť a Presnosť}
Presnosť pri identifikácii procesov, sieťovej aktivity, programov pri štarte. Hodnotenie bezpečnostných funkcií.

\subsection{Návrhy na Vylepšenie}
\begin{itemize}
    \item Odporúčané zmeny alebo doplnky pre budúce verzie.
    \item Nápady na integráciu s inými nástrojmi alebo funkciami.
\end{itemize}



% \subsection{Abbreviations and Acronyms}\label{AA}
% Define abbreviations and acronyms the first time they are used in the text, 
% even after they have been defined in the abstract. Abbreviations such as 
% IEEE, SI, MKS, CGS, ac, dc, and rms do not have to be defined. Do not use 
% abbreviations in the title or heads unless they are unavoidable.

% \subsection{Units}
% \begin{itemize}
% \item Use either SI (MKS) or CGS as primary units. (SI units are encouraged.) English units may be used as secondary units (in parentheses). An exception would be the use of English units as identifiers in trade, such as ``3.5-inch disk drive''.
% \item Avoid combining SI and CGS units, such as current in amperes and magnetic field in oersteds. This often leads to confusion because equations do not balance dimensionally. If you must use mixed units, clearly state the units for each quantity that you use in an equation.
% \item Do not mix complete spellings and abbreviations of units: ``Wb/m\textsuperscript{2}'' or ``webers per square meter'', not ``webers/m\textsuperscript{2}''. Spell out units when they appear in text: ``. . . a few henries'', not ``. . . a few H''.
% \item Use a zero before decimal points: ``0.25'', not ``.25''. Use ``cm\textsuperscript{3}'', not ``cc''.)
% \end{itemize}

% \subsection{Equations}
% Number equations consecutively. To make your 
% equations more compact, you may use the solidus (~/~), the exp function, or 
% appropriate exponents. Italicize Roman symbols for quantities and variables, 
% but not Greek symbols. Use a long dash rather than a hyphen for a minus 
% sign. Punctuate equations with commas or periods when they are part of a 
% sentence, as in:
% \begin{equation}
% a+b=\gamma\label{eq}
% \end{equation}

% Be sure that the 
% symbols in your equation have been defined before or immediately following 
% the equation. Use ``\eqref{eq}'', not ``Eq.~\eqref{eq}'' or ``equation \eqref{eq}'', except at 
% the beginning of a sentence: ``Equation \eqref{eq} is . . .''

% \subsection{\LaTeX-Specific Advice}

% Please use ``soft'' (e.g., \verb|\eqref{Eq}|) cross references instead
% of ``hard'' references (e.g., \verb|(1)|). That will make it possible
% to combine sections, add equations, or change the order of figures or
% citations without having to go through the file line by line.

% Please don't use the \verb|{eqnarray}| equation environment. Use
% \verb|{align}| or \verb|{IEEEeqnarray}| instead. The \verb|{eqnarray}|
% environment leaves unsightly spaces around relation symbols.

% Please note that the \verb|{subequations}| environment in {\LaTeX}
% will increment the main equation counter even when there are no
% equation numbers displayed. If you forget that, you might write an
% article in which the equation numbers skip from (17) to (20), causing
% the copy editors to wonder if you've discovered a new method of
% counting.

% {\BibTeX} does not work by magic. It doesn't get the bibliographic
% data from thin air but from .bib files. If you use {\BibTeX} to produce a
% bibliography you must send the .bib files. 

% {\LaTeX} can't read your mind. If you assign the same label to a
% subsubsection and a table, you might find that Table I has been cross
% referenced as Table IV-B3. 

% {\LaTeX} does not have precognitive abilities. If you put a
% \verb|\label| command before the command that updates the counter it's
% supposed to be using, the label will pick up the last counter to be
% cross referenced instead. In particular, a \verb|\label| command
% should not go before the caption of a figure or a table.

% Do not use \verb|\nonumber| inside the \verb|{array}| environment. It
% will not stop equation numbers inside \verb|{array}| (there won't be
% any anyway) and it might stop a wanted equation number in the
% surrounding equation.

% \subsection{Some Common Mistakes}\label{SCM}
% \begin{itemize}
% \item The word ``data'' is plural, not singular.
% \item The subscript for the permeability of vacuum $\mu_{0}$, and other common scientific constants, is zero with subscript formatting, not a lowercase letter ``o''.
% \item In American English, commas, semicolons, periods, question and exclamation marks are located within quotation marks only when a complete thought or name is cited, such as a title or full quotation. When quotation marks are used, instead of a bold or italic typeface, to highlight a word or phrase, punctuation should appear outside of the quotation marks. A parenthetical phrase or statement at the end of a sentence is punctuated outside of the closing parenthesis (like this). (A parenthetical sentence is punctuated within the parentheses.)
% \item A graph within a graph is an ``inset'', not an ``insert''. The word alternatively is preferred to the word ``alternately'' (unless you really mean something that alternates).
% \item Do not use the word ``essentially'' to mean ``approximately'' or ``effectively''.
% \item In your paper title, if the words ``that uses'' can accurately replace the word ``using'', capitalize the ``u''; if not, keep using lower-cased.
% \item Be aware of the different meanings of the homophones ``affect'' and ``effect'', ``complement'' and ``compliment'', ``discreet'' and ``discrete'', ``principal'' and ``principle''.
% \item Do not confuse ``imply'' and ``infer''.
% \item The prefix ``non'' is not a word; it should be joined to the word it modifies, usually without a hyphen.
% \item There is no period after the ``et'' in the Latin abbreviation ``et al.''.
% \item The abbreviation ``i.e.'' means ``that is'', and the abbreviation ``e.g.'' means ``for example''.
% \end{itemize}
% An excellent style manual for science writers is \cite{b7}.

% \subsection{Authors and Affiliations}\label{AAA}
% \textbf{The class file is designed for, but not limited to, six authors.} A 
% minimum of one author is required for all conference articles. Author names 
% should be listed starting from left to right and then moving down to the 
% next line. This is the author sequence that will be used in future citations 
% and by indexing services. Names should not be listed in columns nor group by 
% affiliation. Please keep your affiliations as succinct as possible (for 
% example, do not differentiate among departments of the same organization).

% \subsection{Identify the Headings}\label{ITH}
% Headings, or heads, are organizational devices that guide the reader through 
% your paper. There are two types: component heads and text heads.

% Component heads identify the different components of your paper and are not 
% topically subordinate to each other. Examples include Acknowledgments and 
% References and, for these, the correct style to use is ``Heading 5''. Use 
% ``figure caption'' for your Figure captions, and ``table head'' for your 
% table title. Run-in heads, such as ``Abstract'', will require you to apply a 
% style (in this case, italic) in addition to the style provided by the drop 
% down menu to differentiate the head from the text.

% Text heads organize the topics on a relational, hierarchical basis. For 
% example, the paper title is the primary text head because all subsequent 
% material relates and elaborates on this one topic. If there are two or more 
% sub-topics, the next level head (uppercase Roman numerals) should be used 
% and, conversely, if there are not at least two sub-topics, then no subheads 
% should be introduced.

% \subsection{Figures and Tables}\label{FAT}
% \paragraph{Positioning Figures and Tables} Place figures and tables at the top and 
% bottom of columns. Avoid placing them in the middle of columns. Large 
% figures and tables may span across both columns. Figure captions should be 
% below the figures; table heads should appear above the tables. Insert 
% figures and tables after they are cited in the text. Use the abbreviation 
% ``Fig.~\ref{fig}'', even at the beginning of a sentence.

% \begin{table}[htbp]
% \caption{Table Type Styles}
% \begin{center}
% \begin{tabular}{|c|c|c|c|}
% \hline
% \textbf{Table}&\multicolumn{3}{|c|}{\textbf{Table Column Head}} \\
% \cline{2-4} 
% \textbf{Head} & \textbf{\textit{Table column subhead}}& \textbf{\textit{Subhead}}& \textbf{\textit{Subhead}} \\
% \hline
% copy& More table copy$^{\mathrm{a}}$& &  \\
% \hline
% \multicolumn{4}{l}{$^{\mathrm{a}}$Sample of a Table footnote.}
% \end{tabular}
% \label{tab1}
% \end{center}
% \end{table}

% \begin{figure}[htbp]
% \centerline{\includegraphics{fig1.png}}
% \caption{Example of a figure caption.}
% \label{fig}
% \end{figure}

% Figure Labels: Use 8 point Times New Roman for Figure labels. Use words 
% rather than symbols or abbreviations when writing Figure axis labels to 
% avoid confusing the reader. As an example, write the quantity 
% ``Magnetization'', or ``Magnetization, M'', not just ``M''. If including 
% units in the label, present them within parentheses. Do not label axes only 
% with units. In the example, write ``Magnetization (A/m)'' or ``Magnetization 
% \{A[m(1)]\}'', not just ``A/m''. Do not label axes with a ratio of 
% quantities and units. For example, write ``Temperature (K)'', not 
% ``Temperature/K''.

% \section*{Acknowledgment}

% The preferred spelling of the word ``acknowledgment'' in America is without 
% an ``e'' after the ``g''. Avoid the stilted expression ``one of us (R. B. 
% G.) thanks $\ldots$''. Instead, try ``R. B. G. thanks$\ldots$''. Put sponsor 
% acknowledgments in the unnumbered footnote on the first page.

% \section*{References}

% Please number citations consecutively within brackets \cite{b1}. The 
% sentence punctuation follows the bracket \cite{b2}. Refer simply to the reference 
% number, as in \cite{b3}---do not use ``Ref. \cite{b3}'' or ``reference \cite{b3}'' except at 
% the beginning of a sentence: ``Reference \cite{b3} was the first $\ldots$''

% Number footnotes separately in superscripts. Place the actual footnote at 
% the bottom of the column in which it was cited. Do not put footnotes in the 
% abstract or reference list. Use letters for table footnotes.

% Unless there are six authors or more give all authors' names; do not use 
% ``et al.''. Papers that have not been published, even if they have been 
% submitted for publication, should be cited as ``unpublished'' \cite{b4}. Papers 
% that have been accepted for publication should be cited as ``in press'' \cite{b5}. 
% Capitalize only the first word in a paper title, except for proper nouns and 
% element symbols.

% For papers published in translation journals, please give the English 
% citation first, followed by the original foreign-language citation \cite{b6}.


\pagebreak
\bibliographystyle{IEEEtran}
\bibliography{literatura}


\end{document}
